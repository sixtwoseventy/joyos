\section{Scoring}

The score that each robot receives is determined by the final state of
the contest table after the match has been played.  Points are scored
for each ball that is over the surface of a scoring area. To count,
the center of the ball must be within the scoring area. Balls must
touch the surface of the table in order to count. Balls stored inside
your robot do not count unless they are touching the table.

There are three scoring areas that are on the wall opposite from the
starting area (black's scoring area is along east wall, white's
scoring area is along the west wall). Small balls are worth 1 point
each, and big balls are worth 3 points each.  The number next to each
scoring area denotes a point multiplier for that area.  For example,
having 2 small balls in each of the scoring areas on the east side
will give the black robot 2*1 + 2*2 + 2*3 = 12 points.

The team with the higher score at the end of the match wins the
match. However, there is one caveat: A robot cannot win unless it has
changed the score in its favor at some point during the
match. Changing the score in your favor can be done by increasing your
own score or decreasing your opponent's score. Therefore if your
opponent scores some balls for you by mistake and you do nothing for
the entire round, you cannot win even if you end up with more points
at the end.

In the event the scores for both robots are the same at the end of the
match, your robot will be awarded win if it has changed the scored in
its favor at some point during the match. Therefore, double-loss and
double-win are possible outcomes. This applies even if the score is
0-0.

If one robot forfeits a match for any reason, it will receive a loss
for that round and will be replaced with a placebo. The match will
then run as it would normally. Therefore, a robot can never win a
match without actually competing.
