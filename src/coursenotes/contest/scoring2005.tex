%\section{Scoring}

The score that each robot receives is determined by the final state of
the contest table after the match has been played.  Points are scored
for each ball that is over the surface of a scoring area. To count,
the center of the ball must be within the scoring area. Balls must
touch the surface of the table in order to count. Balls stored inside
your robot do not count unless they are touching the table.

There are six scoring areas in total. There is one scoring area directly accross from each robot's starting locationand two scoring areas in the oppisit corners from this location. The number of red and green balls will be totalled individually with the final score being determined by the voting total which took place during the round.

The two bins on the sides of the table are for voting, one is for green and the other is for red. At the begining of the round the vote is 0-0. Each ball placed into a voting bin (regardless of the color of the ball) is counted as a vote for for the voting bin's color. At the end of the round the voting determines the value of the balls, with the 'winning' color being worth 2 points and the loosing color being worth -1 points. In the event of a tied vote, the winner will be the color which first reached the tied score, e.g. green has the lead 5-4 and then red ties it 5-5 green is still the winner. The current vote 'winner' will be broadcast to robots throughout the match. At the beginning of the match a 'winner' will be randomly chosen and will remain so until the vote changes from 0-0. 

The team with the higher score at the end of the match wins the
match. However, a robot only wins it has
changed the score in its favor at some point during the
match. Changing the score in your favor can be done by increasing your
own score or decreasing your opponent's score. Therefore if your
opponent scores some balls for you by mistake and you do nothing for
the entire round, you cannot win even if you end up with more points
at the end.

On contest night there will be a score displayed on the big screen for the audience. This score is not {\bf official}; it is simply for audience enjoyment.
 Think of it as the exit polls for our election.

In the event the scores for both robots are the same at the end of the
match, your robot will be awarded win if it has changed the scored in
its favor at some point during the match. Voting alone is not enough, the robot must move balls eiather in or out of one of the scoring areas. Therefore, double-loss and double-win are possible outcomes. This applies even if the score is
0-0.

If one robot forfeits a match for any reason, it will receive a loss
for that round and will be replaced with a placebo. The match will
then run as it would normally. Therefore, a robot can never win a
match without actually competing.
